\documentclass[a4paper,12pt]{article}
\usepackage{graphicx}
\usepackage{verbatim}
\begin{document}
	\title{\textbf{Introduzione alle Copule}}
	\author{Rostagno}
	\date{\today}
	\maketitle

\noindent Le copule sono strumenti matematici che permettono di modellare e stimare la dipendenza tra diverse variabili casuali. Sono particolarmente utili in finanza, dove la dipendenza tra i rendimenti degli asset, i tassi di interesse e i tempi di default sono fattori cruciali per la valutazione del rischio e la determinazione del prezzo di strumenti finanziari complessi. \textbf{L'importanza delle copule risiede nella loro capacità di separare la modellazione delle distribuzioni marginali delle singole variabili dalla modellazione della loro struttura di dipendenza} .

\noindent In altre parole, invece di dover specificare una funzione di distribuzione congiunta per tutte le variabili, è possibile utilizzare una copula per combinare le distribuzioni marginali di ciascuna variabile in una distribuzione congiunta che rifletta la dipendenza desiderata. Questo approccio offre grande flessibilità nella modellazione, poiché consente di scegliere le distribuzioni marginali e la copula in modo indipendente, a seconda delle caratteristiche specifiche dei dati e del problema in esame.

\noindent Ad esempio, si potrebbe utilizzare una distribuzione t di Student per modellare i rendimenti degli indici azionari, che spesso presentano code più spesse rispetto alla distribuzione normale, e quindi utilizzare una copula di Gumbel per rappresentare la dipendenza asimmetrica tra i mercati, con una maggiore probabilità di movimenti congiunti al rialzo rispetto a quelli al ribasso. 

\noindent La teoria delle copule si basa sul teorema di Sklar, che afferma che ogni funzione di distribuzione congiunta può essere espressa in termini di una copula e delle distribuzioni marginali delle variabili. Il teorema di Sklar garantisce l'esistenza e l'unicità della copula nel caso di variabili casuali continue.

\noindent Esistono diverse famiglie di copule, ciascuna con proprietà specifiche in termini di dipendenza di coda, simmetria e altre caratteristiche. Alcune delle famiglie di copule più utilizzate in finanza includono la copula gaussiana, la copula t di Student, le copule Archimedee (come la copula di Gumbel, la copula di Clayton e la copula di Frank) e la copula di Marshall-Olki.

\noindent La scelta della copula più adatta dipende dalla natura del problema e dalle caratteristiche della dipendenza che si desidera modellare. Ad esempio, la copula t di Student è spesso preferita alla copula gaussiana quando si vogliono modellare dipendenze di coda più elevate, mentre le copule Archimedee consentono di modellare diversi tipi di dipendenza asimmetrica.

\noindent Le copule sono strumenti potenti che trovano applicazione in diversi ambiti della finanza, tra cui:

\begin{itemize}
	\item Pricing di opzioni multivariate e altri derivati: le copule possono essere utilizzate per modellare la dipendenza tra i sottostanti di un'opzione basket, un'opzione rainbow o altri derivati multi-asset
	, consentendo una valutazione più accurata del prezzo di questi strumenti.
	\item Gestione del rischio: le copule sono ampiamente utilizzate nella modellazione del rischio di credito, dove consentono di stimare la probabilità di default congiunta di diverse attività o controparti. Le copule sono anche utilizzate nella stima del Value at Risk (VaR) di portafogli contenenti attività con distribuzioni non normali e dipendenze complesse.
	\item Calibrazione e simulazione: la flessibilità delle copule consente di calibrare i modelli ai dati di mercato in modo efficiente e di simulare scenari di mercato realistici che tengano conto della dipendenza tra le variabili.
\end{itemize}


\noindent In sintesi, le copule rappresentano uno strumento matematico versatile e potente per la modellazione della dipendenza in finanza, con un ampio spettro di applicazioni pratiche nella valutazione del rischio, nel pricing di derivati e nella gestione del portafogli.

\section*{\textbf{Fondamenti Matematici delle Copule}}

Le copule sono strumenti matematici che permettono di modellare e rappresentare la dipendenza tra variabili casuali. A differenza di misure di dipendenza tradizionali come la correlazione lineare, le copule catturano la dipendenza in modo più completo, includendo la dipendenza nelle code delle distribuzioni e non limitandosi a relazioni lineari.

\noindent Ecco una spiegazione delle formule e delle proprietà chiave: \\

\noindent \textbf{Definizione di Copula} \\ 

\noindent Una d-copula è una funzione $C: [0,1]^d \rightarrow [0,1]$, dove $d \geq 2$ (numero di variabili, nelle proprietà seguenti consideriamo le copule bivariate), che soddisfa le seguenti proprietà:

\begin{enumerate}
	\item  \textbf{Groundedness:} \\ $C(u, 0) = C(0, v) = 0$ per ogni $u, v \in [0,1]^2$. Ciò significa che la copula è zero se una delle variabili è zero.
	\item \textbf{Marginalità:} \\ $C(u, 1) = u$, $C(1, v) = v$ per ogni $u, v \in [0,1]^2$.  Questa proprietà assicura che la copula sia coerente con le distribuzioni marginali, ovvero che quando una delle variabili assume il suo valore massimo, la copula coincide con la funzione di ripartizione dell'altra variabile.
	\item \textbf{2-crescita (o 2-increasing):} \\  $C(u_2, v_2) - C(u_2, v_1) - C(u_1, v_2) + C(u_1, v_1) \geq 0$ per ogni $u_1 \leq u_2$, $v_1 \leq v_2$ in $[0,1]^2$. Questa proprietà assicura che la copula sia non decrescente in entrambe le variabili, il che è necessario affinché la copula rappresenti una dipendenza positiva o negativa tra le variabili.
\end{enumerate}

\noindent \textbf{Teorema di Sklar} \\ 

\noindent Questo teorema, centrale nella teoria delle copule, stabilisce un legame tra le copule e le funzioni di distribuzione congiunta. \\
In breve, il teorema afferma che:

\noindent Data una funzione di distribuzione congiunta $F(x, y)$ con marginali $F_1(x)$ e $F_2(y)$, esiste una copula $C$ tale che:

$$F(x, y) = C(F_1(x), F_2(y))$$

\noindent Inoltre, se $F_1(x)$ e $F_2(y)$ sono continue, allora la copula $C$ è unica. \\

\noindent \textbf{Conseguenze del Teorema di Sklar:}
\begin{itemize}
	\item Costruzione di modelli di dipendenza: Permette di costruire una funzione di distribuzione congiunta a partire da distribuzioni marginali arbitrarie e da una copula che ne modella la dipendenza. Questa proprietà è particolarmente utile per modellare dati reali, dove spesso si conoscono le distribuzioni marginali ma non la struttura di dipendenza.
	\item Separazione tra marginali e dipendenza: Mette in luce come la struttura di dipendenza tra le variabili sia completamente catturata dalla copula, indipendentemente dalle distribuzioni marginali.\\[2\baselineskip]
\end{itemize}

\noindent \textbf{Famiglie di copule} \\

\noindent Esistono diverse famiglie di copule, classificate in base alla loro struttura o ai metodi utilizzati per la loro costruzione. Di seguito, vengono elencate alcune delle principali famiglie:

\renewcommand{\labelitemi}{}
\begin{itemize}
	\item \textbf{Fréchet-Hoeffding:} Questa famiglia include le copule che rappresentano i limiti inferiore ($W$) e superiore ($M$) della dipendenza tra due variabili casuali. La copula $W$ rappresenta la perfetta dipendenza negativa, mentre la copula $M$ rappresenta la perfetta dipendenza positiva
	\item \textbf{Cuadras-Augé:}  Questa famiglia di copule è costruita come una media geometrica ponderata delle copule $M$ e $P$, dove $P$ rappresenta l'indipendenza tra le variabili
	\item \textbf{Marshall-Olkin:}  Questa famiglia di copule è spesso utilizzata per modellare la dipendenza tra variabili casuali che rappresentano tempi di vita.
	\item \textbf{Archimedee:} Queste copule sono generate da una funzione detta "generatore". Le copule Archimedee sono popolari per la loro flessibilità e la relativa facilità di utilizzo
\end{itemize}

\noindent \textbf{Proprietà delle copule} \\

\noindent Le copule posseggono diverse proprietà che le rendono utili per la modellazione della dipendenza. Alcune di queste proprietà sono:
\renewcommand{\labelitemi}{\textbullet}
\begin{itemize}
	\item \textbf{Invarianza rispetto a trasformazioni monotone crescenti:} Le copule sono invarianti rispetto a trasformazioni strettamente crescenti delle variabili marginali.
	\item \textbf{Misure di concordanza:} Diverse misure di concordanza come la rho di Spearman e la tau di Kendall possono essere espresse in termini di copule.
	\item \textbf{Dipendenza di coda:} Le copule possono catturare la dipendenza tra le code delle distribuzioni marginali, ovvero la tendenza delle variabili ad assumere valori estremi congiuntamente.
\end{itemize}

\noindent Possiamo quindi affermare che le copule offrono un approccio potente e flessibile per la modellazione della dipendenza tra variabili casuali. La loro capacità di separare la struttura di dipendenza dalle distribuzioni marginali, la loro invarianza rispetto a trasformazioni monotone crescenti e la loro capacità di catturare la dipendenza di coda, le rendono strumenti preziosi in molte applicazioni pratiche.

\newpage

\begin{center}
	\Huge \textbf{Applicazioni Finanziarie dei Modelli di Copula} 
\end{center}

\vspace{1cm}
\noindent\textbf{Superiorità delle Copule sulla Correlazione Lineare con Distribuzioni Non Normali} \\

\noindent L'assunzione di normalità dei rendimenti, tipico di modelli come quello di Black-Scholes, è spesso disatteso nei mercati finanziari. Le serie storiche di strumenti come azioni ed obbligazioni dimostrano la presenza di code più pesanti rispetto a quanto previsto dalla distribuzione normale e la diffusione di prodotti derivati con payoff non lineari amplifica ulteriormente questo fenomeno.  

\noindent In questo contesto, la \textbf{correlazione lineare}, misurata ad esempio con il coefficiente di Pearson, si dimostra uno strumento limitato.  Essa cattura solo le dipendenze lineari tra le variabili, mentre le relazioni tra gli asset finanziari possono assumere forme ben più complesse. La correlazione lineare è efficace solo quando le variabili sono legate da relazioni lineari. Tuttavia, in presenza di legami non lineari, la correlazione lineare potrebbe essere fuorviante. Ad esempio, una variabile con distribuzione chi-quadrato è perfettamente correlata al suo quadrato, che ha distribuzione normale, ma la correlazione lineare non sarebbe in grado di rappresentare correttamente questa relazione. \\

\noindent Spieghiamo il significato del coefficente di Pearson.\\

\noindent \textbf{Definizione:} Il coefficiente di correlazione lineare, noto anche come correlazione di Pearson, è una misura della dipendenza lineare tra due variabili casuali che assumono valori reali e che hanno varianza finita. È definito come la covarianza delle due variabili divisa per il prodotto delle loro deviazioni standard.
Formalmente, per due variabili casuali non degeneri $X$ e $Y$ appartenenti a $L^2$ , il coefficiente di correlazione lineare $\rho_{XY}$ è:\\
\[
\rho_{XY} = \frac{\textit{cov(X,Y)}}{\sqrt{var(X)var(Y)}}
\]

\noindent Il coefficiente di correlazione di Pearson assume valori compresi tra -1 e +1, dove:
\begin{itemize}
	\item +1 indica una perfetta correlazione lineare positiva: all'aumentare di una variabile, l'altra aumenta in modo perfettamente lineare.
	\item -1 indica una perfetta correlazione lineare negativa: all'aumentare di una variabile, l'altra diminuisce in modo perfettamente lineare.
	\item 0 indica l'assenza di correlazione lineare: non c'è una relazione lineare tra le due variabili.\\
\end{itemize}

\noindent È importante sottolineare che il coefficiente di correlazione di Pearson misura solo la dipendenza lineare. Due variabili possono essere fortemente dipendenti in modo non lineare e avere comunque un coefficiente di correlazione di Pearson pari a zero.\\

\noindent Le \textbf{copule}, invece, offrono un approccio più flessibile per modellare la dipendenza tra variabili casuali, anche in presenza di distribuzioni non normali. Il vantaggio principale delle copule risiede nella loro capacità di separare la struttura di dipendenza dalle distribuzioni marginali. Questo permette di combinare diverse distribuzioni marginali, capaci di cogliere la non-normalità dei rendimenti (ad esempio la distribuzione t di Student o distribuzioni asimmetriche), con una vasta gamma di copule che descrivono la struttura di dipendenza. 

\noindent Il Teorema di Sklar, fondamento della teoria delle copule, afferma che ogni funzione di distribuzione congiunta può essere espressa in termini di distribuzioni marginali e di una copula. Questa proprietà permette di costruire modelli di dipendenza altamente flessibili, adatti a rappresentare le complesse relazioni tra gli asset finanziari. Ad esempio, è possibile utilizzare una copula gaussiana per modellare la struttura di dipendenza, pur mantenendo distribuzioni marginali non gaussiane per i singoli asset. In questo modo, si ottiene un modello in grado di catturare sia la non-normalità dei rendimenti che la struttura di dipendenza tra gli stessi.

\noindent In definitiva, le copule rappresentano uno strumento più completo e affidabile rispetto alla correlazione lineare per modellare le dipendenze tra variabili casuali, soprattutto in presenza di distribuzioni non normali. La loro flessibilità e capacità di rappresentare accuratamente le complesse relazioni tra gli asset le rendono essenziali per una corretta valutazione del rischio, un pricing accurato e una migliore comprensione delle dinamiche dei mercati finanziari. 
\newline

\noindent \textbf{Dipendenze di Coda e la loro Importanza nel VaR e nell'Expected Shortfall}\\

\noindent La dipendenza di coda si riferisce alla tendenza di due o più variabili casuali a muoversi insieme in modo più estremo nelle code delle loro distribuzioni, rispetto a quanto previsto da una distribuzione normale con la stessa correlazione lineare. In altre parole, la dipendenza di coda misura la probabilità che si verifichino eventi estremi congiuntamente.\\

\noindent Sappiamo che la non normalità a livello univariato è associata al cosiddetto problema della \textit{fat-tail}. In un contesto
multivariato, il problema della \textit{fat-tail} può essere riferito sia alle distribuzioni
marginali univariate che alle distribuzioni congiunte di probabilità di grandi movimenti di mercato. Questo concetto è chiamato \textit{tail dependence}. L'uso di funzioni copula ci permette di modellare separatamente queste due caratteristiche.
Per rappresentare la dipendenza dalla coda consideriamo la probabilità che un evento con
probabilità inferiore a v si verifichi nella prima variabile, dato che un evento con probabilità
inferiore a v si verifica nella seconda. In concreto, ci chiediamo quale sia la probabilità di
osservare, ad esempio, un crollo con probabilità inferiore di v=1\% nell'indice Nikkei 225,
dato che nell'indice S\&P 500 si è verificato un crollo con probabilità inferiore all'1\%. Si ha

\[
\lambda(v) \equiv \Pr(Q_{NKY} \leq v \mid Q_{SP} \leq v)
\]
\[
= \frac{\Pr(Q_{NKY} \leq v, Q_{SP} \leq v)}{\Pr(Q_{SP} \leq v)}
\]
\[
= \frac{C(v,v)}{v}
\]
\newline
\noindent \textbf{Tipi di dipendenza di coda:}\\

\noindent Dopo che abbiamo calcolato il nostro $\lambda(v)$, possiamo dividere la dipendenza di coda in due tipi principali:
\begin{itemize}
	\item \textbf{Dipendenza di coda inferiore:} misura la probabilità che entrambe le variabili assumano valori estremamente bassi contemporaneamente.\\
	\[
	\lambda_L \equiv \lim_{v \to 0^+} \frac{C(v, v)}{v}
	\]
	\item \textbf{Dipendenza di coda superiore:} misura la probabilità che entrambe le variabili assumano valori estremamente alti contemporaneamente.\\
	\[
	\lambda_U = \lim_{v \to 1^-} \lambda_v \equiv \lim_{v \to 1^-} \frac{\Pr(\overline{Q}_{NKY} > v, \overline{Q}_{SP} > v)}{\Pr(\overline{Q}_{SP} > v)}
	\]
	\[
	= \lim_{v \to 1^-} \frac{1 - 2v + C(v, v)}{1 - v}
	\]
	\newline
	\newline
\end{itemize}

\noindent \textbf{Importanza nel VaR e nell'Expected Shortfall:}\\

\noindent La dipendenza di coda ha un impatto significativo sul calcolo del VaR e dell'Expected Shortfall, due misure di rischio ampiamente utilizzate nella gestione del rischio finanziario.

\begin{itemize}
	\item \textbf{VaR (Value at Risk):} rappresenta la perdita massima stimata che un portafoglio potrebbe subire in un determinato orizzonte temporale e con un dato livello di confidenza.
	\item \textbf{Expected Shortfall:} rappresenta la perdita media attesa in caso di superamento del VaR.
\end{itemize}

\noindent In presenza di dipendenza di coda, il VaR e l'Expected Shortfall calcolati assumendo una distribuzione normale tendono a sottostimare il rischio effettivo del portafoglio. Questo perché la distribuzione normale non riesce a catturare adeguatamente la probabilità di eventi estremi congiunti. Mentre utilizzando le copule per modellare la dipendenza tra gli asset di un portafoglio, è possibile ottenere una stima più accurata del VaR e dell'Expected Shortfall, tenendo conto della probabilità di eventi estremi congiunti. Quindi grazie alle copule si ottiene una stima più accurata del rischio di portafoglio e si possono prendere decisioni di investimento più consapevoli.
\newline

\noindent \textbf{Tariffare le opzioni}\\

\noindent La valutazione di opzioni multivariate, come le opzioni basket o rainbow, che dipendono da più attività sottostanti, rappresenta una sfida significativa in finanza. Le copule forniscono un potente strumento per affrontare questo problema.
\begin{itemize}
	\item In sostanza, una copula viene utilizzata per costruire la distribuzione congiunta dei prezzi delle attività sottostanti alla scadenza.
	\item Questa distribuzione viene quindi utilizzata per calcolare il valore atteso del payoff dell'opzione in base a tutti i possibili risultati dei prezzi delle attività sottostanti.
	\item Attualizzando questo valore atteso al tasso privo di rischio, si ottiene il prezzo dell'opzione.
\end{itemize}

\noindent Nei mercati incompleti, dove non esiste una misura di probabilità unica priva di arbitraggio, le copule sono fondamentali per derivare strategie di super-replicazione. Queste strategie mirano a creare un portafoglio di attività negoziabili che replichi il payoff dell'opzione in tutte le possibili situazioni future, garantendo così l'assenza di opportunità di arbitraggio. Le copule consentono di determinare i limiti superiori e inferiori del prezzo dell'opzione in base alle diverse ipotesi sulla struttura di dipendenza tra le attività sottostanti.
Mostriamo ora degli esempi:
\begin{itemize}
	\item \textbf{Opzioni arcobaleno:} queste opzioni, che dipendono dal minimo o dal massimo di un paniere di attività, possono essere valutate utilizzando le copule per catturare la dipendenza tra i rendimenti delle attività. Le fonti dimostrano come le copule possano essere utilizzate per derivare strategie di super-replicazione per le opzioni arcobaleno, fornendo limiti superiori e inferiori al prezzo.
	\item \textbf{Opzioni barriera:} per le opzioni in cui l'esercizio è condizionato al fatto che il prezzo dell'attività sottostante raggiunga o meno una determinata barriera, le copule possono essere utilizzate per modellare la dipendenza tra il percorso del prezzo dell'attività e l'evento di attivazione della barriera.
\end{itemize}

\noindent \textbf{Gestione dei rischi}\\

\noindent Le copule possono essere utilizzate per modellare la dipendenza tra diversi tipi di rischio, come rischio di mercato, rischio di credito e rischio operativo. Ciò è particolarmente utile per le istituzioni finanziarie che sono esposte a più tipi di rischio, in quanto consente loro di valutare il rischio complessivo a cui sono esposte. Ad esempio, una banca può utilizzare le copule per modellare la dipendenza tra le insolvenze sui prestiti e i movimenti dei tassi di interesse, consentendo loro di valutare il rischio del proprio portafoglio prestiti in diversi scenari economici.

\noindent In particolare, nella gestione del rischio di credito, le copule vengono utilizzate nella valutazione di strumenti di debito strutturati come le obbligazioni garantite da crediti (CDO). Le CDO sono obbligazioni garantite da un pool di attività sottostanti, come mutui o prestiti alle imprese. Il rischio di credito di una CDO dipende dalla dipendenza tra le insolvenze delle attività sottostanti. Le copule forniscono un modo flessibile per modellare questa dipendenza, consentendo agli investitori di valutare il rischio e il rendimento delle CDO in modo più accurato.\\
\newline
\textbf{Tipi di modelli di copula}\\

\begin{itemize}
	\item Copula gaussiana: descrive la dipendenza tra variabili casuali utilizzando la distribuzione normale multivariata. Non è in grado di catturare la dipendenza dalla coda. È definita come la funzione di distribuzione congiunta di un vettore normale multivariato standard, dove ogni variabile marginale è stata trasformata nella sua forma uniforme standard utilizzando la funzione di distribuzione normale standard inversa.
	\[
	C_R^{Ga}(u,v)=\Phi_R (\Phi^{-1}(u), \Phi^{-1}(v)) 
	\] 
	\[
	= \int_{-\infty}^{\Phi^{-1}(u)} \int_{-\infty}^{\Phi^{-1}(v)} \frac{1}{2\pi \sqrt{1 - \rho_{XY}^2}} 
	\exp \left( \frac{2\rho_{XY}st - s^2 - t^2}{2(1 - \rho_{XY}^2)} \right) ds\, dt
	\]
	\item Copula t di Student: Questa copula può catturare la dipendenza dalla coda e viene spesso utilizzata per modellare i rendimenti degli asset che mostrano code pesanti. La copula t di Student bivariata è data dalla seguente formula:\\ $C_v^{t}(u,v)= \int_{-\infty}^{t_v^{-1}(u)} \int_{-\infty}^{t_v^{-1}(v)} \frac{\Gamma(\frac{v+2}{2})}{ \Gamma(\frac{v}{2}) \pi v \sqrt{1-\rho^2}} (1+\frac{x^2-2\rho xy + y^2}{v(1-\rho^2)})^{-\frac{v+2}{2}} dxdy$ dove $t_v^{-1}$ è l'inversa della funzione di distribuzione t di Student univariata con $v$ gradi di libertà e $\rho$ è il coefficiente di correlazione.
	\item Copula di Clayton: questa copula mostra una forte dipendenza dalla coda inferiore, il che significa che le variabili hanno maggiori probabilità di assumere insieme valori estremi bassi.È esaustiva e fornisce la copula del prodotto se $\alpha=0$, il limite inferiore di Frechet $max(v+z-1,0)$ quando $\alpha=-1$ e quello superiore per $\alpha \to +\infty$. È definita dalla seguente formula: $C(v,z)=max[(v^{-\alpha}+z^{-\alpha}-1)^{-\frac{1}{\alpha}},0]$.
	\item Copula di Gumbel: questa copula mostra una forte dipendenza dalla coda superiore, indicando che le variabili hanno maggiori probabilità di assumere insieme valori estremi elevati. Fornisce la copula del prodotto se $\alpha=1$ e il limite superiore di Frechet $min(v,z)$ per $\alpha \to +\infty$. È definita dalla seguente formula: $C(v,z)=exp{-[(-\ln v)^\alpha + (-\ln z)^\alpha]^{\frac{1}{\alpha}} }$.
	\item Copula di Frank: questa copula è una copula simmetrica che può catturare sia la dipendenza dalla coda superiore che quella dalla coda inferiore. Si riduce alla copula del prodotto se $\alpha=0$ e raggiunge i limiti inferiore e superiore di Frechet rispettivamente per $\alpha \to -\infty$ e $\alpha \to +\infty$. È definita dalla seguente formula: $C(v,z)=-\frac{1}{\alpha}\ln (\frac{1+(e^{-\alpha v}-1)(e^{-\alpha z}-1)}{e^{-\alpha}-1})$.
\end{itemize}

\begin{center}
	\Huge \textbf{Preparazione dei dati e assunzioni} 
\end{center}

\vspace{1cm}
\noindent\textbf{3.1 Raccolta dei dati} \\

\noindent Il seguente dataset contiene dati storici sul DAX, che rappresentano prezzi o indici di mercato rilevanti. La raccolta dati fornisce i valori di Open, High, Low, Close partendo dal giorno 02/01/2020 ore 01:15:00 e terminando con il giorno 03/03/2022 ore 09:14:00. La raccolta e la gestione adeguata di tali dati sono fondamentali per analizzare le dipendenze tra strumenti finanziari tramite modelli di copula. Tramite questi dati calcoleremo i rendimenti giornalieri, un passaggio necessario per la modellazione delle dipendenze.\\
\begin{figure}[h] % L'ambiente figure permette di gestire il posizionamento
	\centering % Centra l'immagine
	\includegraphics[width=0.8\textwidth]{dfDax.png} % Inserisce l'immagine con larghezza metà pagina
	\caption{Inizio e fine del dataset} % Aggiunge una didascalia all'immagine
	\label{fig:immagine} % Aggiunge un'etichetta per riferimenti interni
\end{figure}
\newpage
\noindent\textbf{3.2 Pulizia e pre-elaborazione dei dati} \\

\noindent I dati finanziari spesso includono anomalie come valori mancanti o outlier che devono essere gestiti prima dell'analisi. Saranno implementate le seguenti tecniche di pulizia dei dati:

\begin{itemize}
	\item \textbf{Gestione dei valori mancanti:} rimuovere eventuali righe con valori mancanti per evitare distorsioni.
	\item \textbf{Gestione degli outlier:} utilizzare tecniche di filtraggio per identificare ed eliminare gli outlier, assicurando che l'analisi si concentri sui valori centrali più rappresentativi.
\end{itemize}
\textbf{Outlier:} 
nel contesto dei dati azionari, un outlier (o valore anomalo) è un'osservazione che si discosta significativamente dalla norma o dalla tendenza generale del dataset. In altre parole, si tratta di un dato che è molto diverso rispetto agli altri valori presenti nel campione.\\

\noindent Il motivo per cui vanno eliminati è che possono distorcere le analisi statistiche, poiché influenzano la media, la deviazione standard e altre misure di dispersione dei dati.\\

\noindent Ecco alcune righe di codice utilizzate per "pulire" i dati:
\begin{verbatim}
	import pandas as pd
	
	# Load the data
	data = pd.read_csv('DAX_3Y-1M.csv',index_col='DateTime',
	parse_dates=True)
	
	# Drop rows with missing values
	data=data.dropna()
	
	# Verifica e gestione degli outlier tramite interquartile
	range (IQR)
	Q1 = data.quantile(0.25)
	Q3 = data.quantile(0.75)
	IQR = Q3 - Q1
	filtered_data = data[~((data < (Q1 - 1.5 * IQR)) | 
	(data > (Q3 + 1.5 * IQR))).any(axis=1)]
\end{verbatim}
Circa 6000 righe di dati sono state eliminate da questo processo.\\

\noindent\textbf{3.3 Trasformazione dei dati} \\

\noindent Dopo la pulizia, è necessario trasformare i dati per ottenere una scala uniforme. Poiché i modelli di copula richiedono margini uniformi, trasformeremo i dati in rendimenti logaritmici per ottenere stazionarietà e calcoleremo i punteggi standardizzati:\\

\noindent Ecco alcune righe di codice utilizzate per "trasformare" i dati:
\begin{verbatim}
	import numpy as np
	
	# Calcolo dei rendimenti logaritmici
	
	log_returns = np.log(filtered_data / 
	filtered_data.shift(1)).dropna()
	
	# Standardizzazione dei dati
	(sottraggo la media e divido per la deviazione standard):
	
	standardized_data = (log_returns - log_returns.mean())
	/ log_returns.std()
\end{verbatim}

\begin{figure}[h] % L'ambiente figure permette di gestire il posizionamento
	\centering % Centra l'immagine
	\includegraphics[width=0.8\textwidth]{dfnormal.png} % Inserisce l'immagine con larghezza metà pagina
	\caption{Rendimenti logaritmici standardizzati} % Aggiunge una didascalia all'immagine
	\label{fig:immagine} % Aggiunge un'etichetta per riferimenti interni
\end{figure}
\newpage
\noindent\textbf{3.4 Normalizzazione} \\

\noindent Per applicare correttamente i modelli di copula, i dati devono essere trasformati in una distribuzione uniforme sull'intervallo $[0,1]$. Questo passaggio permette ai dati di adattarsi meglio alla funzione di copula che verrà utilizzata per modellare le dipendenze:\\
\begin{verbatim}
	from scipy.stats import norm
	
	# Normalizzazione tramite la funzione di 
	distribuzione cumulativa (CDF)
	
	uniform_data = norm.cdf(standardized_data)
\end{verbatim}

\begin{figure}[h] % L'ambiente figure permette di gestire il posizionamento
	\centering % Centra l'immagine
	\includegraphics[width=0.8\textwidth]{norm.png} % Inserisce l'immagine con larghezza metà pagina
	\caption{Dati uniformati all'intervallo [0,1]} % Aggiunge una didascalia all'immagine
	\label{fig:immagine} % Aggiunge un'etichetta per riferimenti interni
\end{figure}

\noindent\textbf{3.5 Assunzioni nei Modelli di Copula } \\

\noindent Per l'uso corretto dei modelli di copula, è essenziale discutere alcune assunzioni chiave:\\

\begin{itemize}
	\item \textbf{Uniformità marginale:} l'assunzione primaria nella modellazione delle copule è che le variabili marginali abbiano distribuzioni uniformi. Abbiamo utilizzato la funzione di distribuzione cumulativa per garantire questa uniformità.
	\item \textbf{Struttura di dipendenza:} i modelli di copula modellano la struttura di dipendenza tra le variabili senza fare ipotesi sui margini. Ciò significa che, dopo la trasformazione, possiamo utilizzare diversi tipi di copule per analizzare come i vari strumenti finanziari si muovono insieme.
	\item \textbf{Stazionarietà:} è importante che i dati siano stazionari, ovvero che le loro proprietà statistiche (come media e varianza) siano costanti nel tempo. Abbiamo utilizzato la differenziazione logaritmica per rendere i dati stazionari.
	\item \textbf{Normalità:} per l'utilizzo di una copula Gaussiana, i margini devono approssimare la normalità. Sebbene non sia strettamente necessario per altre copule come la t-Copula, una trasformazione per avvicinarsi alla normalità può essere utile per semplificare l'analisi.
\end{itemize}

\noindent\textbf{3.6 Distribuzione dei rendimenti} \\

\begin{figure}[h] % L'ambiente figure permette di gestire il posizionamento
	\centering % Centra l'immagine
	\includegraphics[width=0.8\textwidth]{rendimenti.png} % Inserisce l'immagine con larghezza metà pagina
	\label{fig:immagine} % Aggiunge un'etichetta per riferimenti interni
\end{figure}

\noindent Ecco la distribuzione dei rendimenti logaritmici del DAX, calcolata su un subset dei dati disponibili. La distribuzione mostra la tipica forma a campana, con alcune code più pesanti, suggerendo la presenza di eventi estremi più frequenti rispetto a una normale distribuzione Gaussiana. Questa caratteristica supporta l'idea di utilizzare copule come la Student-t, che meglio cattura queste dipendenze nelle code.\\
\newpage
\noindent Di seguito la matrice di correlazione tra i rendimenti logaritmici delle diverse colonne di prezzo (Open, High, Low, Close) sempre del subset.

\begin{figure}[h] % L'ambiente figure permette di gestire il posizionamento
	\centering % Centra l'immagine
	\includegraphics[width=0.8\textwidth]{corr.png} % Inserisce l'immagine con larghezza metà pagina
	\caption{Matrice di correlazione} % Aggiunge una didascalia all'immagine
	\label{fig:immagine} % Aggiunge un'etichetta per riferimenti interni
\end{figure}

\noindent\textbf{3.7 Assunzioni generali}\\

\noindent Considerando i dati analizzati e i risultati ottenuti dalle elaborazioni, possiamo formulare le seguenti assunzioni per l'applicazione dei modelli di copula:

\begin{enumerate}
	\item \textbf{Uniformità marginale}\\
	Una delle assunzioni principali per applicare i modelli di copula è che le variabili marginali siano uniformi. Nel nostro caso, i rendimenti logaritmici calcolati per le diverse variabili (Open, High, Low, Close) sono stati trasformati in modo tale da poter essere considerati approssimativamente stazionari, ma non sono ancora stati resi uniformi. Per l'applicazione delle copule, sarà necessario trasformare i rendimenti normalizzati in una distribuzione uniforme sull'intervallo [0,1], ad esempio usando la funzione di distribuzione cumulativa empirica. Questo passaggio garantisce che le dipendenze tra variabili siano modellate correttamente senza distorsioni derivanti dalle distribuzioni marginali.
	\item \textbf{Stazionarietà dei dati}\\
	Per applicare correttamente i modelli di copula, è necessario che le serie temporali siano stazionarie, ovvero che le proprietà statistiche come media e varianza siano costanti nel tempo. Abbiamo trasformato i prezzi in rendimenti logaritmici per ottenere una serie più stazionaria rispetto ai dati originali di prezzo. Tuttavia, è possibile che ci siano ancora componenti non stazionarie, come trend residui o stagionalità, che potrebbero influire sui risultati. La verifica e il trattamento di eventuali residui non stazionari sono fondamentali per garantire la validità dei modelli di copula.
	\item \textbf{Struttura di dipendenza}\\
	L'analisi dei rendimenti logaritmici ha mostrato una correlazione positiva tra le variabili, sebbene con valori differenti per ciascuna coppia (ad esempio, correlazione relativamente più alta tra High e Close, e più bassa tra Open e Close). Questa osservazione implica che esiste una struttura di dipendenza tra le variabili di prezzo, che va oltre la correlazione lineare. Le copule ci permetteranno di catturare meglio questa dipendenza, soprattutto nei casi in cui le correlazioni sono condizionate da situazioni estreme (code pesanti).
	\item \textbf{Dipendenze di coda}\\
	Osservando la distribuzione dei rendimenti logaritmici, è evidente che la distribuzione presenta code più pesanti rispetto a una normale distribuzione Gaussiana. Questo suggerisce una maggiore probabilità di eventi estremi (sia positivi che negativi), specialmente durante periodi di volatilità del mercato. Pertanto, è ragionevole assumere che le variabili presentino dipendenze di coda, rendendo modelli come la copula di Student-t o la copula Clayton adatti per catturare le correlazioni nelle code inferiori, particolarmente durante i ribassi di mercato.
	\item \textbf{Non-normalità delle distribuzioni marginali}\\
	I rendimenti logaritmici non seguono una distribuzione normale; piuttosto, mostrano asimmetria e code più pesanti. Sebbene la copula Gaussiana possa essere utilizzata per una prima analisi, è preferibile utilizzare copule come la Student-t per gestire deviazioni dalla normalità, particolarmente utili per modellare le code e le correlazioni durante gli eventi estremi.
	\item \textbf{Asimmetria nelle dipendenze}\\
	La matrice di correlazione calcolata tra i rendimenti (Open, High, Low, Close) mostra differenze nei livelli di correlazione tra le diverse variabili. Questa asimmetria nelle correlazioni suggerisce che alcuni modelli di copula, come la Clayton (per le code inferiori) o la Gumbel (per le code superiori), potrebbero fornire una descrizione più accurata delle dipendenze, rispetto a modelli simmetrici come la copula Gaussiana.
	\item \textbf{Condizioni di diversificazione}\\
	La copula di Frank potrebbe essere adatta per modellare le dipendenze tra le variabili che non mostrano comportamenti particolarmente forti nelle code (ovvero, dipendenze moderate e stabili). Tuttavia, i dati indicano la presenza di code pesanti, quindi questa copula potrebbe essere utilizzata solo come confronto con modelli che catturano meglio le dipendenze estreme.
\end{enumerate}

\noindent\textbf{Conclusione sulle assunzioni}\\

\noindent Uniformità e stazionarietà sono requisiti fondamentali per l'applicazione dei modelli di copula. I dati sono stati trasformati per soddisfare parzialmente queste assunzioni.\\
Dipendenze di coda e non-normalità suggeriscono l'uso di copule robuste come la Student-t o modelli asimmetrici come la Clayton o la Gumbel per catturare meglio le relazioni tra variabili durante condizioni di stress di mercato.\\
La asimmetria delle correlazioni evidenziata dalla matrice di correlazione indica la necessità di copule che possano gestire differenti tipi di dipendenze nelle code.\\
Queste assunzioni ci permettono di scegliere il modello di copula più appropriato per analizzare le dipendenze tra i rendimenti del DAX e comprendere meglio il comportamento del mercato in diverse condizioni economiche.




\end{document}